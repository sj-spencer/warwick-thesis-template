% !TeX root = ../../Thesis.tex
\chapter{Effects of laser bandwidth on inflationary stimulated Raman scattering}
\label{chp:broadbandSRS}

\section{Motivation and literature review}
\section{1D simulations and threshold}
\section{2D simulations}
%\subsection{Han Wen Talk 13/01/20}
%\begin{enumerate}
%    \item studying bandwidth suppression of iSRS in a density gradient by forcing a single mode
%    \item Sim params: Te = 3keV; linear density profile 0.06-0.18; seeded so the resonant density is approx 0.12nCr, klD approx 0.33; immobile ions.
 %  \item Step 1: ran with Te = 0.1keV and recovered Rosenbluth gain (approx fluid limit)
 %   \item Step 2: Convective gain theory works well at low intensity, even with large kLd
 %   \item Step 3: I0 = 5e14 W/cm2; Iseed = 6e-4 I0. In this case you see growth from noise as well as from seed, not sufficiently foriced/driven.
%    \item Step 4: increase seed level Iseed = 3e-2 I0. The resonant point for the seed shifts to higher densities as inflation happens.
%    \item Step 5: SSD type bandwidth was applied to the pump 
    $$E_{pump} = 0.5E_0\left(exp[i(\omega_0t-kx+\sum_MA_Msin(\omega_Mt))]\right)$$
%\end{enumerate}
%words innit

