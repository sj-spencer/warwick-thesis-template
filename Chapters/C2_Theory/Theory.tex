% !TeX root = ../../Thesis.tex
\chapter{Theory}
\label{chp:theory}

This chapter aims to equip the reader with the tools they need to read, and to engage critically with, the rest of the thesis. By the end of this chapter, the reader should be familiar with the following concepts and methods:
\begin{compactenum}
	\item Stimulated Raman scattering (\acrshort{SRS}) as a three-wave parametric instability.
	\item The difference between convective and absolute SRS, and how we can quantify their gain.
	\item The difference between fluid and kinetic SRS, and how we can use $\kld$ to delineate these two regimes.
	\item The effect of large trapped-particle populations on SRS, including the growth of additional resonant modes.
\end{compactenum}

%%%%%%%%%%%%%%%%%%%%%%%%%%%%%%%%%%%%%%%%%%
% START OF PREREQUISITES
%%%%%%%%%%%%%%%%%%%%%%%%%%%%%%%%%%%%%%%%%%

\section{Pre-requisites}
This section provides a minimal set of pre-requisites I will be assuming in this chapter. We start with Maxwell's equations in differential form:

\begin{subequations}\label{eqn:Maxwell}
\begin{alignat}{2}
 &  \nabla \times \vec{E} = -\frac{\partial \vec{B}}{\partial t} && \text{\,\,\,\,\,\,\,\,\,(Faraday-Lenz Law)} \\
 & \nabla \times \vec{B} = \mu_{0} \vec{j}+\frac{1}{c^{2}} \frac{\partial \vec{E}}{\partial t} && \text{\,\,\,\,\,\,\,\,\,(Ampere-Maxwell Law)}\\
 & \nabla \cdot \vec{B} = 0 && \text{\,\,\,\,\,\,\,\,\,(No Magnetic Monopoles)}\\
 & \nabla \cdot \vec{E} = \frac{\rho}{\epsilon_{0}}  && \text{\,\,\,\,\,\,\,\,\,(Gauss's Law)}.
\end{alignat}
\end{subequations} These describe the evolution of the vector fields $[\vec{E}(\vec{x},t),\vec{B}(\vec{x},t)]$ using the mathematical notation $\nabla = \left(\frac{\partial}{\partial x},\frac{\partial}{\partial y},\frac{\partial}{\partial z} \right)$ to denote the scalar function divergence $(\nabla \cdot)$ and the vector function curl $(\nabla \times)$. The charge density $\rho$ and current density $\vec{j}$ are most properly calculated by summing the contributions of the exact phase-space coordinates of every particle in the plasma. However, we can make an approximation and assume that that the phase space is continuous and take an average to get the distribution function $f(\vec{x},\vec{v},t)$. We can now define $\rho$ and $\vec{j}$ in terms of the distribution function like so:

\begin{equation}
\rho= q \int d \vec{v} f(\vec{x},\vec{v},t)
\end{equation}

\begin{equation}
\vec{j} = q \int d \vec{v} \vec{v} f(\vec{x},\vec{v},t).
\end{equation}

In order to close the Maxwell equations, we need an equation which tells us how the distribution function evolves in terms of $[\vec{E}(\vec{x},t),\vec{B}(\vec{x},t)]$. Following any course in the kinetic theory of plasmas (for example, \citet{MMathPhys}), we end up with the Vlasov equation for a collisionless plasma:

\begin{equation}
 	\frac{\partial f}{\partial t}+\vec{v} \cdot {\nabla} f+\frac{q}{m}\left(\vec{E}+\frac{\vec{v} \times \vec{B}}{c}\right) \cdot \frac{\partial f}{\partial \vec{v}}=0
\end{equation} By taking moments, with respect to velocity, of the Vlasov equation (and moving to 1D for brevity), we obtain the following reduced description of the plasma in terms of an electron fluid: 
\begin{equation}
\frac{\partial n}{\partial t}+\frac{\partial}{\partial x}(n u)=0 \text{\,\,\,\,\,\,\,\,\,(0th moment: Continuity equation)}
\label{eqn:continuity}
\end{equation}

\begin{equation}
m_e n \left(\frac{\partial u}{\partial t} + u\frac{\partial}{\partial x}u\right)=-\frac{\partial P}{\partial x}+q E_{x} n \text{\,\,\,\,\,\,\,\,\,(1st moment: Momentum equation)}.
\label{eqn:momentum}
\end{equation} Where $n$ is the number density (calculated by taking the zeroth moment of the distribution function); $u$ is the macroscopic flow velocity (calculated by taking the first moment of the distribution function); and $P$ is a scalar pressure. In order to close this set of equations, we require an expression for the scalar pressure $P$ that does not introduce any higher moments of the distribution function. Typically in this thesis the timescales are such that thermal conduction can be neglected, and we assume that the process is adiabatic. This means that our final equation takes the form:

\begin{equation}
\frac{d}{dt}\left(\frac{P}{n^{\gamma}}\right) = 0 \text{\,\,\,\,\,\,\,\,\,(Energy equation)},
\label{eqn:energy}
\end{equation} with $\gamma = C_p/C_v = 1 + 2/n = 3$ in one dimension. When taken together, Equations \ref{eqn:continuity}, \ref{eqn:momentum}, and \ref{eqn:energy} are known as the hydrodynamic equations of motion. For a multi-species plasma, each species will have its own continuity, momentum, and energy equations.

We recall the fundamental time and length scales associated with a plasma: the plasma frequency and the Debye length. The plasma frequency describes the natural frequency of oscillation of electrons in a cold plasma, its expression in terms of the electron number density can be derived by considering the displacement of a block of plasma and considering the restoring forces acting on it, to give:

\begin{equation}
	\omega_{pe} = \sqrt{\frac{n_e e^2}{m_e \epsilon_0}}.
\end{equation} This fundamental time-scale allows us to define a fundamental length-scale based on how far a particle with thermal velocity $v_{\text{th}}$ can travel in one oscillation period.

\begin{equation}
	\lambda_D = \frac{v_{\mathrm{th}}}{\omega_{pe}} \propto \left(\frac{\mathrm{temperature}}{\mathrm{density}}\right)^{1/2}.
\end{equation}\label{eqn:debye} The thermal velocity is given by:
\begin{equation}
	v_{\text{th}} = \sqrt{\frac{\gamma k_BT_e}{m_e}},
\end{equation}\label{eqn:vth} where $k_B$ is the Boltzmann constant.

The final pre-requisite assumed in this chapter is the concept of \textit{linearising} a system of equations. Given a system of equations, we can often make the system easier to solve by assuming that the variables can be written in the form: $n = n_0 + n_1$, where $n_0$ is an initial or equilibrium value and $|n_1| \ll |n_0|$ is a small perturbation to this value. We then substitute this new expression for the variable back into the system of equations to understand the response of the system to this small perturbation. Typically, linearising means neglecting, in your equations, quadratic or higher order terms in the perturbation variables. 

%%%%%%%%%%%%%%%%%%%%%%%%%%%%%%%%%%%%%
% END OF PREREQUISITES
%%%%%%%%%%%%%%%%%%%%%%%%%%%%%%%%%%%%%

%%%%%%%%%%%%%%%%%%%%%%%%%%%%%%%%%%%%%
% START OF FLUID THEORY
%%%%%%%%%%%%%%%%%%%%%%%%%%%%%%%%%%%%

\section{Fluid theory of three-wave instabilities}
Equipped with this brief revision of the fundamental relationships which describe a plasma, we can now move onto the theory behind our object of study: stimulated Raman scattering (\acrshort{SRS}).

\subsection{Dispersion curves and linear modes}\label{subsec:dispersion}

Equipped with the hydrodynamic equations of motion (Equations \ref{eqn:continuity}, \ref{eqn:momentum}, and \ref{eqn:energy}), we now look for normal modes of the system. These describe oscillations in which all variables vary sinusoidally with the same frequency, meaning that they can be written as $A(x,t) = A_1(x)e^{-i\omega t}$. A non-magnetised, thermal, two-species (electron-ion) plasma supports two longitudinal electrostatic modes and one electromagnetic mode. An excellent derivation of these modes can be found in Chapter Three of \citet{Jones1985}.

By linearising the electron fluid equations and Maxwell's equations, assuming that ion motion is negligible, and looking for solutions which are harmonic in space and time ($n_e,P,u_e,E \propto  e^{i(kx -\omega t)}$)  we can derive the dispersion relation for electron plasma waves (\acrshort{EPW}s):
\begin{equation}\label{eqn:EPW}
	\omega^{2}=\omega_{p e}^{2}+3 k^{2} v_{\text{th}}^{2}. 
\end{equation} Electron plasma waves are also known as ``Langmuir waves'' after Irving Langmuir, and Equation \ref{eqn:EPW} is sometimes called the ``Bohm-Gross dispersion relation''.

By a similar linear analysis, this time assuming that $T_e \gg T_i$ and that $\omega \ll \omega_{pi}$ (the ion plasma frequency), we find the dispersion relation for ion acoustic waves (\acrshort{IAW}s):
\begin{equation}\label{eqn:IAW}
 \omega =  c_{s} k,
\end{equation} where $c_{s}$ is the sound speed,
\begin{equation}
c_s = \sqrt{\frac{Z\gamma_ek_BT_e + \gamma_ik_BT_i}{m_i}}.
\end{equation} $Z$ is the ion charge state, and $\gamma_e,\gamma_i$ are the ratio of specific heats for electron and ion fluids, respectively. Typically, we take $\gamma_i = 3$ which corresponds to 1D adiabatic compression of the ions, and $\gamma_e = 1$, corresponding to isothermal compression of the electrons.

Finally, we consider the problem of a high frequency electromagnetic wave (\acrshort{EMW}) travelling in a plasma. The dispersion relation for electromagnetic waves in a plasma is:
\begin{equation}\label{eqn:EMW}
 \omega^{2}=\omega_{p e}^{2}+c^2k^{2}.
\end{equation} This expression shows that the frequency of EMWs in a plasma is bounded below by the electron plasma frequency. The limit $\omega_0=\omega_{pe}$ defines a critical density past which EMWs cannot propagate, since the the electrons shield can respond to shield out the field of the EMW:
\begin{equation}\label{eqn:ncrit}
 n_{cr} = \sqrt{\frac{n_e e^2}{m_e\epsilon_0}}.
\end{equation} 

Write about WKB ? \todo{Should I write something about the WKB solutions since our corona isn't homogeneous?}

Figure \ref{fig:EMW_EPW_disp} shows graphically the dispersion relations for electron plasma waves (\acrshort{EPW}s) and electromagnetic waves (\acrshort{EMW}s) in a non-magnetised, thermal plasma. The angular frequency of the waves is normalised to the electron plasma frequency, and the wavenumber to the inverse Debye length. The black dashed line $\omega / k = c$ is the dispersion relation for \acrshort{EMW} in vacuum, and dashed line $\omega / k = \sqrt{3/2}v_{th}$ is the electron-acoustic wave (\acrshort{EAW}) for a population of hot electrons with thermal velocity $v_{th}$. The frequency of the electron-acoustic wave satisfies $\omega_{pi} < \omega < \omega_{pe}$. The \acrshort{EPW} branch and electron acoustic branch actually merge at short wavelengths ($\kld\sim 0.53$) \citep{Rose2003}. The electron-acoustic wave (\acrshort{EAW}) and its cousin the beam-acoustic mode (\acrshort{BAM}) will be covered in Section \ref{sec:kineticTheory}, where we look at the limits of the hydrodynamic theory and consider some important additions from kinetic theory.


\begin{figure}[ht]
   \centering
    \includegraphics[width=0.9\columnwidth]{Chapters/C2_Theory/EPW_EMW_dispersion.pdf}
    \caption{Dispersion relations for electromagnetic waves and electron plasma waves in a non-magnetised, hot plasma with $T_e= 5\text{keV}$, $n_e=0.1n_{cr}$. The limits are chosen such that $\kld < 0.5$ for the \acrshort{EPW}s, which is required for Equation \ref{eqn:EPW} to be valid.}
    \label{fig:EMW_EPW_disp}
\end{figure}{}




\subsection{Three-wave instabilities}
From linear analysis of the hydrodynamic equations in an un-magnetised plasma (Equations \ref{eqn:continuity}, \ref{eqn:momentum}, \ref{eqn:energy}) we have recovered three waves: the \acrshort{EPW}, \acrshort{IAW}, and \acrshort{EMW}. Linear analysis, however, does not tell us anything about how the perturbations associated with one of the waves might affect one or more of the other waves present in the plasma. A second order analysis, which considers contributions proportional to the product of two oscillating quantities, is required to understand how these waves might influence each other. A three-wave instability is the lowest-order example of this non-linear interaction.


We assume small non-linear coupling, which allows us to say that each of the three waves has angular frequency and wave-number ($w,k$) which locally satisfy the linear dispersion relations presented in Section \ref{subsec:dispersion}. Furthermore, we insist that frequency and wave-number matching are satisfied:




\begin{figure}[ht]
   \centering
   \fbox{
    \includegraphics[width=0.75\columnwidth]{Chapters/C2_Theory/3wave_instabilities.pdf}}
    \caption{This is them innit}
    \label{fig:3wave_instabilities}
\end{figure}{}


 In the linear analysis, we looked for solutions for the oscillating quantities of the form: $A(x,t) = A_1\text{exp}(i(\omega t - kx))$. When we consider the non-linear terms

With these three waves, we find the following three-wave parametric instabilities:

%\begin{alignat*}{2}
%	 	&\text{EMW1} \rightarrow \text{EPW} + \text{EMW2} &\text{\,\,\,\,\,\,\,\,\,(Stimulated Raman Scattering)}\\
%	 	&\text{EMW1} \rightarrow \text{IAW} + \text{EMW2} &\text{\,\,\,\,\,\,\,\,\,(Stimulated Brillouin Scattering)}\\
%	 	&\text{EMW1} \rightarrow \text{EPW1} + \text{EPW2} &\text{\,\,\,\,\,\,\,\,\,(Two-Plasmon Decay)}\\
%	 	&\text{EPW1} \rightarrow \text{EPW2} + \text{IAW} &\text{\,\,\,\,\,\,\,\,\,(Langmuir Decay Instability)}\\
%	 	&\text{EMW} \rightarrow \text{EPW} + \text{IAW} &\text{\,\,\,\,\,\,\,\,\,(The Parametric Instability)}\\
%\end{alignat*}



\subsection{Stimulated Raman scattering}

Stimulated Raman scattering can grow in two ways: 

\subsubsection{Rosenbluth gain}

Since the instability we are concerned with is convective, we would like to understand what the maximum wave amplitude is for the daughter waves, according to the linear theory, in order to determine how SRS will grow in the fluid regime.

The derivation below follows the the steps laid out in \citet{Nishikawa1976}, with the following adapatations for this thesis:some steps written out in more explicit detail; notation changes, to improve the readability; and minor typographical corrections. 

We use our physical understanding of the system make the following assumptions:
\begin{enumerate}
	\item undamped EMW $\Gamma_1 = 0$
	\item strong damping and slow convection of EPW
	\item constant source at 0, maximum value at $+\infty$
	\item perfect matching at $x=0$, assume kappa is well-approximated by 
	$\kappa(x) = \kappa'(0)x$.
\end{enumerate}

Consider a three-wave parametric instability that takes place in a plasma slab with a density gradient in $x$ with a uniform pump. The density gradient leads to $x$-varying wavenumbers for the waves, so we define the `wavenumber mismatch' as $\kappa = k_0(x) - k_1(x) - k_2(x)$, where perfect matching is defined by the condition $\kappa(x=0) = 0$ and we insist that $\kappa = \kappa' x$. The daughter waves can be described by the following pair of partial differential equations:

\begin{equation}
 \left(\frac{\partial}{\partial t} + v_1\frac{\partial}{\partial x} + \Gamma_1 \right)a_1 = \gamma_0a_2\text{exp}\left(\frac{i\kappa'x^2}{2}\right)
\end{equation}

\begin{equation}
 \left(\frac{\partial}{\partial t} + v_2\frac{\partial}{\partial x} + \Gamma_2 \right)a_2 = \gamma_0a_1\text{exp}\left(\frac{-i\kappa'x^2}{2}\right);
\end{equation} 
where $\Gamma_{1,2}$ are the damping rates; $a_{1,2}$ the action amplitudes; and $v_{1,2}$ the group velocities of the two waves. 

WHAT ARE WE TRYING TO DO, WHAT MOTIVATES THIS TRANSFORM?

Recalling the definition of the Laplace transform of a function $f(t)$: $L\{f(t)\}= F(p) = \int_0^\infty e^{-pt} f(t) dt$ we take the Laplace transform of these equations to get






\subsection{Stimulated Brillouin scattering}
Since SBS is only important in Chapter \ref{chp:magSRS}, we will only give a very brief overview of it here.

%%%%%%%%%%%%%%%%%%%%%%%%%%%%%%%%%%%%%%%%%%%%
% END OF FLUID THEORY
%%%%%%%%%%%%%%%%%%%%%%%%%%%%%%%%%%%%%%%%%%%%

%%%%%%%%%%%%%%%%%%%%%%%%%%%%%%%%%%%%%%%%%%%
% START OF KINETIC THEORY
%%%%%%%%%%%%%%%%%%%%%%%%%%%%%%%%%%%%%%%%%%%


\section{Kinetic theory of three-wave instabilities}\label{sec:kineticTheory}

Kinetic theory takes account of the complete distribution function of particle velocities, 

\subsection{Landau damping}


Unfortunately, by looking for normal modes, we have forced ourselves to find normal modes and we have ignored the possibility of damped oscillations in the system. Considering the electron plasma wave as an initial value problem 
A key aspect of quasilinear theory is its identification of the distinction between reso-
nant and non-resonant particles, scattering and diffusion \citep{Sagdeev2018}.

\subsection{Nonlinear frequency shift}
non-linear basis for trapping induced SRS modes found in \cite{Rose2001}


\subsection{Inflationary stimulated Raman scattering}

Stimulated Raman scattering (\acrshort{SRS}) which takes place in the kinetic regime is often called inflationary SRS (\acrshort{iSRS}), and we will use this shorthand throughout the thesis. The name ``inflationary SRS" refers to the ``inflation" of measures of SRS-activity (such as the reflectivity) above the level predicted by fluid theory, often by many orders of magnitude. The kinetic effect responsible for this inflation depends on whether SRS occurs in a homogeneous or inhomogeneous plasma.

For the case of an inhomogeneous plasma below the quarter critical density, SRS grows as a 




\subsection{Autoresonance}

\begin{figure}[ht]
    \centering
    \includegraphics[width=0.8\columnwidth]{Chapters/C2_Theory/AR_diagnostic.pdf}
    \caption{Example of autoresonant growth in an EPOCH simulation with parameters: $n_{min} = 0.06 n_{\text{crit}}$; $n_{max} = 0.17 n_{\text{crit}}$; $T_e = 4.5\si{keV}$; $\text{nPPC}=10,000$; $I_0 = 2 \times 10^{15}\si{\watt / \centi\metre^2}$. Black dashed line comes from Chapman \textit{et al.} \citep{Chapman2012} formula.}
    \label{fig:AR_diagnostic}
\end{figure}{}

%%%%%%%%%%%%%%%%%%%%%%%%%%%%%%%%%%%%%%%%%%%%%%%
% END OF KINETIC THEORY
%%%%%%%%%%%%%%%%%%%%%%%%%%%%%%%%%%%%%%%%%%%%%%%

%%%%%%%%%%%%%%%%%%%%%%%%%%%%%%%%%%%%%%%%%%%%%%%
% START OF MAGNETIC FIELD 
%%%%%%%%%%%%%%%%%%%%%%%%%%%%%%%%%%%%%%%%%%%%%%%

\section{Three-wave instabilities in a magnetic field}

In a magnetic field, we can define a new fundamental frequency. The cyclotron frequency is given by:
\begin{equation}
\omega_c = \frac{eB_0}{m_e}.
\end{equation}

The three waves which were supported by the un-magnetised plasma: electron-plasma wave; ion-acoustic wave; and electromagetic wave; are modified by an external magnetic field. The modified \acrshort{EPW} is the upper hybrid wave (\acrshort{UHW})

The dispersion relation for the upper-hybrid 

\section{Final remarks}
In this chapter, we presented results from fluid theory, the dispersion relations, the results of considering higher order terms in the fluid or kinetiic theory of waves in a plasma, and kinetic effects on the SRS EPW. In the SRS literature referenced in this thesis, there is a significant confusion of language. For example, many papers refer to the ``linear theory of SRS'', referring to the theory derived from linearising the fluid equations. However, it is clear that since SRS involves a coupling of three waves, it is a non-linear process. From this point on, we will use the phrases ``fluid theory of SRS'' and ``kinetic theory of SRS'' to delineate the regimes where SRS is well-described without considering trapped particle effects (often called the libnear theory of SRS).

%\bibliographystyle{plainnat}
%\bibliography{Chapters/C2_Theory/Theory}
