% !TeX root = ../../Thesis.tex
\chapter{Theory}
\label{chp:theory}

This chapter aims to equip the reader with the tools they need to read, and to engage critically with, the rest of the thesis. By the end of this chapter, the reader should be familiar with the following concepts and methods:
\begin{enumerate}
	\item Stimulated Raman scattering (\acrshort{SRS}) as a three-wave parametric instability.
	\item The difference between convective and absolute SRS, and how we can quantify their gain.
	\item The difference between fluid and kinetic SRS, and how we can use $\kld$ to delineate these two regimes.
	\item The effect of large trapped-particle populations on SRS, including the growth of additional resonant modes.
\end{enumerate}


\section{Pre-requisites}
This section provides a minimal set of pre-requisites I will be assuming in this chapter. First up, we have Maxwell's equations in differential form:



\begin{equation}
\nabla \times \vec{E}=-\frac{\partial \vec{B}}{\partial t} \text{\,\,\,\,\,\,\,\,\,(Faraday-Lenz Law)}
\end{equation}

\begin{equation}
\nabla \times \vec{B}=\mu_{0} \vec{j}+\frac{1}{c^{2}} \frac{\partial \vec{E}}{\partial t}
\text{\,\,\,\,\,\,\,\,\,(Ampere-Maxwell Law)}
\end{equation}

\begin{equation}
\nabla \cdot \vec{B}=0 \text{\,\,\,\,\,\,\,\,\,(No Magnetic Monopoles)}
\end{equation}

\begin{equation}
\nabla \cdot \vec{E}=\frac{\rho}{\epsilon_{0}} \text{\,\,\,\,\,\,\,\,\,(Gauss's Law)}.
\end{equation}


These describe the evolution of the vector fields $[\vec{E}(\vec{x},t),\vec{B}(\vec{x},t))]$ using the mathematical notation $\nabla = \left(\frac{\partial}{\partial x},\frac{\partial}{\partial y},\frac{\partial}{\partial z} \right)$ to denote the scalar function divergence $(\nabla \cdot)$ and the vector function curl $(\nabla \times)$. The charge density $\rho$ and current density $\vec{j}$ are calculated by taking moments of the plasma distribution function like so:

\begin{equation}
\rho= q \int d \vec{v} f
\end{equation}

\begin{equation}
\vec{j} = q \int d \vec{v} \vec{v} f.
\end{equation}

Another important variable we can calculate from  the plasma distribution function is the thermal velocity of the plasma. 


Next we have two of the characteristic features of a hot plasma: the plasma frequency and the Debye length. The plasma frequency describes the natural frequency of oscillation of (generally) electrons in a cold plasma, its expression in terms of the electron number density can be derived by considering the displacement of a block of plasma and considering the restoring forces, to give:

\begin{equation}
	\omega_{pe} = \sqrt{\frac{n_e e^2}{m_e \epsilon_0}}.
\end{equation}

This fundamental time-scale allows us to define a fundamental length-scale based on how far a particle with thermal velocity $v_{\text{th}}$ can travel in one oscillation period.

\begin{equation}
	\lambda_D \equiv \frac{v_{\mathrm{th}}}{\omega_{pe}} \propto \left(\frac{\mathrm{temperature}}{\mathrm{density}}\right)^{1/2}
\end{equation}\label{eqn:debye}

\section{The basics: three-wave parametric instabilities}
Equipped with this brief revision of the fundamental relationships which describe a plasma, we can now move onto the theory of

\subsection{Dispersion curves and linear modes}

A non-magnetised plasma supports two electrostatic (no time-varying magnetic fields) modes and one electromagnetic mode.

\subsection{Landau damping}

\subsection{Stimulated Raman scattering}


\section{Non-linear effects}
A key aspect of quasilinear theory is its identification of the distinction between reso-
nant and non-resonant particles, scattering and diffusion \citep{Sagdeev2018}.

\subsection{Nonlinear frequency shift}
non-linear basis for trapping induced SRS modes found in \cite{Rose2001}
\subsection{Nonlinear Landau damping}



\subsection{Rosenbluth gain for SRS}

Since the instability we are concerned with is convective, we would like to understand what the maximum wave amplitude is for the daughter waves, according to the linear theory, in order to determine how SRS will grow in the fluid regime.

The derivation below follows the the steps laid out in \citet{Nishikawa1976}, with the following adapatations for this thesis:some steps written out in more explicit detail; notation changes, to improve the readability; and minor typographical corrections. 

We use our physical understanding of the system make the following assumptions:
\begin{enumerate}
	\item undamped EMW $\Gamma_1 = 0$
	\item strong damping and slow convection of EPW
	\item constant source at 0, maximum value at $+\infty$
	\item perfect matching at $x=0$, assume kappa is well-approximated by 
	$\kappa(x) = \kappa'(0)x$.
\end{enumerate}

Consider a three-wave parametric instability that takes place in a plasma slab with a density gradient in $x$ with a uniform pump. The density gradient leads to $x$-varying wavenumbers for the waves, so we define the `wavenumber mismatch' as $\kappa = k_0(x) - k_1(x) - k_2(x)$, where perfect matching is defined by the condition $\kappa(x=0) = 0$ and we insist that $\kappa = \kappa' x$. The daughter waves can be described by the following pair of partial differential equations:

\begin{equation}
 \left(\frac{\partial}{\partial t} + v_1\frac{\partial}{\partial x} + \Gamma_1 \right)a_1 = \gamma_0a_2\text{exp}\left(\frac{i\kappa'x^2}{2}\right)
\end{equation}

\begin{equation}
 \left(\frac{\partial}{\partial t} + v_2\frac{\partial}{\partial x} + \Gamma_2 \right)a_2 = \gamma_0a_1\text{exp}\left(\frac{-i\kappa'x^2}{2}\right);
\end{equation} 
where $\Gamma_{1,2}$ are the damping rates; $a_{1,2}$ the action amplitudes; and $v_{1,2}$ the group velocities of the two waves. 

WHAT ARE WE TRYING TO DO, WHAT MOTIVATES THIS TRANSFORM?

Recalling the definition of the Laplace transform of a function $f(t)$: $L\{f(t)\}= F(p) = \int_0^\infty e^{-pt} f(t) dt$ we take the Laplace transform of these equations to get






\section{Autoresonance}

\begin{figure}[ht]
    \centering
    \includegraphics[width=0.8\columnwidth]{Chapters/C2_Theory/AR_diagnostic.pdf}
    \caption{Example of autoresonant growth in an EPOCH simulation with parameters: $n_{min} = 0.06 n_{\text{crit}}$; $n_{max} = 0.17 n_{\text{crit}}$; $T_e = 4.5\si{keV}$; $\text{nPPC}=10,000$; $I_0 = 2 \times 10^{15}\si{\watt / \centi\metre^2}$. Black dashed line comes from Chapman \textit{et al.} \citep{Chapman2012} formula.}
    \label{fig:AR_diagnostic}
\end{figure}{}

%\bibliographystyle{plainnat}
%\bibliography{Chapters/C2_Theory/Theory}
