% !TeX root = ../../Thesis.tex
\chapter{Theory}
\label{chp:theory}

Remember to refer to Notes for Cours Ecole Doctorale, Advanced Theory of Plasmas, Ecole Polytechnique Fédérale de Lausanne (Saved on Mendeley)

\section{Motivation}
\subsection{Kinetic plasma instabilities}
\subsection{Three-wave parametric instabilities}

\section{Linear theory}
The aim of the linear theory of \acrshort{SBS} is to understand how the growth
 of the instability depends on factors such as: plasma density; laser intensity; and temperature.
\subsection{Dispersion curves and linear modes}
\subsection{Zoology/topology of modes}
\subsubsection{Beam acoustic modes}
\subsubsection{Electron acoustic modes}
ie on the omega,k diagram how many things are there that we should be worried
about? BGK modes, B
\subsection{Landau damping}

\section{Quasi-linear theory}
%\subsection{Landau damping}

\section{Nonlinear theory}
\subsection{Three-wave parametric instabilities}
\subsection{Nonlinear frequency shift}
\subsection{Nonlinear Landau damping}
non-linear basis for trapping induced SRS modes found in \cite{Rose2001}, also
 a thing called the nonlinear dielectric function



ohm-Gross waves; Beam acoustic modes, other acoustic modes.
etc. Where are they and how do they relate? Ie the Bohm-Gross and Electron
Acoustic modes join at klD~0.53

\section{Autoresonance}
