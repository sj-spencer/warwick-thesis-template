% !TeX root = ../../Thesis.tex
\chapter{Introduction}
\label{chp:introduction}
The material presented in this thesis represents kinetic simulations of laser-plasma interactions (\acrshort{LPI}) relevant to direct-drive inertial confinement fusion (\acrshort{ICF}). In this introductory chapter, an overview of the goals of, and approaches to, inertial confinement fusion is given; including specific details of the `shock-ignition' (\acrshort{SI}) ICF scheme. We then review the previous work concerning laser-plasma interactions in direct-drive ICF, and motivate the use of kinetic modelling throughout this thesis. Finally, we offer an outline of the rest of the thesis.

\section{Inertial Confinement Fusion}
\subsection{Aims}
aim 1: be part of the future energy mmix, provide baseline energy for when renewables are fluctuating \citep{Nicholas2021}
\subsection{Direct and indirect drive ICF}
\subsection{Shock-ignition}
test\citep{Ribeyre2009}


\section{Previous work}
\subsection{Laser-plasma interactions in shock-ignition}
\subsection{Simulations}
\subsubsection{Kinetic models}
\subsubsection{Fluid models with kinetic effects}
Basically it's interesting that you can put some of these effects into a fluid model, such as here \citep{Tran2020}
\subsection{Experiments}

\subsection{Why PIC in this thesis?}
Question: doesn't PIC suck for predicting experimental observables? 

\section{Thesis Outline}
Gonna write these things

%\bibliographystyle{plainnat}
%\bibliography{Chapters/C1_Introduction/Introduction}
